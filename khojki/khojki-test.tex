\documentclass[12pt]{article}
\usepackage{xunicode,xltxtra,fontspec}

\parindent=0pt

\newfontfamily{\khoj}[Renderer=Graphite]{KhojkiGraphite}

\setmainfont{Times New Roman}


\begin{document}

Test file for Khojki Graphite \\ 
Anshuman Pandey <pandey@umich.edu> \\
12 November 2015
\vspace{.25in}

This file must be processed with \XeLaTeX{}.
\bigskip

Normative sequences of base consonant + vowel signs, anusvara, 
virama, nukta, and shadda. Sequences of consonant + 
\{vowel sign i, vowel sign ii, vowel sign u\} 
are rendered as ligatures. 
\bigskip

{\Large\khoj 
𑈈 𑈈𑈬 𑈈𑈭 𑈈𑈮 𑈈𑈯 𑈈𑈰 𑈈𑈱 𑈈𑈲 𑈈𑈳 𑈈𑈴 𑈈𑈵 𑈈𑈶 𑈈𑈷 \\ 

𑈉 𑈉𑈬 𑈉𑈭 𑈉𑈮 𑈉𑈯 𑈉𑈰 𑈉𑈱 𑈉𑈲 𑈉𑈳 𑈉𑈴 𑈉𑈵 𑈉𑈶 𑈉𑈷 \\ 

𑈊 𑈊𑈬 𑈊𑈭 𑈊𑈮 𑈊𑈯 𑈊𑈰 𑈊𑈱 𑈊𑈲 𑈊𑈳 𑈊𑈴 𑈊𑈵 𑈊𑈶 𑈊𑈷 \\ 

𑈋 𑈋𑈬 𑈋𑈭 𑈋𑈮 𑈋𑈯 𑈋𑈰 𑈋𑈱 𑈋𑈲 𑈋𑈳 𑈋𑈴 𑈋𑈵 𑈋𑈶 𑈋𑈷 \\ 

𑈌 𑈌𑈬 𑈌𑈭 𑈌𑈮 𑈌𑈯 𑈌𑈰 𑈌𑈱 𑈌𑈲 𑈌𑈳 𑈌𑈴 𑈌𑈵 𑈌𑈶 𑈌𑈷 \\ 

𑈍 𑈍𑈬 𑈍𑈭 𑈍𑈮 𑈍𑈯 𑈍𑈰 𑈍𑈱 𑈍𑈲 𑈍𑈳 𑈍𑈴 𑈍𑈵 𑈍𑈶 𑈍𑈷 \\ 

𑈎 𑈎𑈬 𑈎𑈭 𑈎𑈮 𑈎𑈯 𑈎𑈰 𑈎𑈱 𑈎𑈲 𑈎𑈳 𑈎𑈴 𑈎𑈵 𑈎𑈶 𑈎𑈷 \\ 

𑈏 𑈏𑈬 𑈏𑈭 𑈏𑈮 𑈏𑈯 𑈏𑈰 𑈏𑈱 𑈏𑈲 𑈏𑈳 𑈏𑈴 𑈏𑈵 𑈏𑈶 𑈏𑈷 \\ 

𑈐 𑈐𑈬 𑈐𑈭 𑈐𑈮 𑈐𑈯 𑈐𑈰 𑈐𑈱 𑈐𑈲 𑈐𑈳 𑈐𑈴 𑈐𑈵 𑈐𑈶 𑈐𑈷 \\ 

𑈑 𑈑𑈬 𑈑𑈭 𑈑𑈮 𑈑𑈯 𑈑𑈰 𑈑𑈱 𑈑𑈲 𑈑𑈳 𑈑𑈴 𑈑𑈵 𑈑𑈶 𑈑𑈷 \\ 

% 𑈒 𑈒𑈬 𑈒𑈭 𑈒𑈮 𑈒𑈯 𑈒𑈰 𑈒𑈱 𑈒𑈲 𑈒𑈳 𑈒𑈴 𑈒𑈵 𑈒𑈶 𑈒𑈷 \\ 

𑈓 𑈓𑈬 𑈓𑈭 𑈓𑈮 𑈓𑈯 𑈓𑈰 𑈓𑈱 𑈓𑈲 𑈓𑈳 𑈓𑈴 𑈓𑈵 𑈓𑈶 𑈓𑈷 \\ 

𑈔 𑈔𑈬 𑈔𑈭 𑈔𑈮 𑈔𑈯 𑈔𑈰 𑈔𑈱 𑈔𑈲 𑈔𑈳 𑈔𑈴 𑈔𑈵 𑈔𑈶 𑈔𑈷 \\ 

𑈕 𑈕𑈬 𑈕𑈭 𑈕𑈮 𑈕𑈯 𑈕𑈰 𑈕𑈱 𑈕𑈲 𑈕𑈳 𑈕𑈴 𑈕𑈵 𑈕𑈶 𑈕𑈷 \\ 

𑈖 𑈖𑈬 𑈖𑈭 𑈖𑈮 𑈖𑈯 𑈖𑈰 𑈖𑈱 𑈖𑈲 𑈖𑈳 𑈖𑈴 𑈖𑈵 𑈖𑈶 𑈖𑈷 \\ 

𑈗 𑈗𑈬 𑈗𑈭 𑈗𑈮 𑈗𑈯 𑈗𑈰 𑈗𑈱 𑈗𑈲 𑈗𑈳 𑈗𑈴 𑈗𑈵 𑈗𑈶 𑈗𑈷 \\ 

𑈘 𑈘𑈬 𑈘𑈭 𑈘𑈮 𑈘𑈯 𑈘𑈰 𑈘𑈱 𑈘𑈲 𑈘𑈳 𑈘𑈴 𑈘𑈵 𑈘𑈶 𑈘𑈷 \\ 

𑈙 𑈙𑈬 𑈙𑈭 𑈙𑈮 𑈙𑈯 𑈙𑈰 𑈙𑈱 𑈙𑈲 𑈙𑈳 𑈙𑈴 𑈙𑈵 𑈙𑈶 𑈙𑈷 \\ 

𑈚 𑈚𑈬 𑈚𑈭 𑈚𑈮 𑈚𑈯 𑈚𑈰 𑈚𑈱 𑈚𑈲 𑈚𑈳 𑈚𑈴 𑈚𑈵 𑈚𑈶 𑈚𑈷 \\ 

𑈛 𑈛𑈬 𑈛𑈭 𑈛𑈮 𑈛𑈯 𑈛𑈰 𑈛𑈱 𑈛𑈲 𑈛𑈳 𑈛𑈴 𑈛𑈵 𑈛𑈶 𑈛𑈷 \\ 

𑈜 𑈜𑈬 𑈜𑈭 𑈜𑈮 𑈜𑈯 𑈜𑈰 𑈜𑈱 𑈜𑈲 𑈜𑈳 𑈜𑈴 𑈜𑈵 𑈜𑈶 𑈜𑈷 \\ 

𑈝 𑈝𑈬 𑈝𑈭 𑈝𑈮 𑈝𑈯 𑈝𑈰 𑈝𑈱 𑈝𑈲 𑈝𑈳 𑈝𑈴 𑈝𑈵 𑈝𑈶 𑈝𑈷 \\ 

𑈞 𑈞𑈬 𑈞𑈭 𑈞𑈮 𑈞𑈯 𑈞𑈰 𑈞𑈱 𑈞𑈲 𑈞𑈳 𑈞𑈴 𑈞𑈵 𑈞𑈶 𑈞𑈷 \\ 

𑈟 𑈟𑈬 𑈟𑈭 𑈟𑈮 𑈟𑈯 𑈟𑈰 𑈟𑈱 𑈟𑈲 𑈟𑈳 𑈟𑈴 𑈟𑈵 𑈟𑈶 𑈟𑈷 \\ 

𑈠 𑈠𑈬 𑈠𑈭 𑈠𑈮 𑈠𑈯 𑈠𑈰 𑈠𑈱 𑈠𑈲 𑈠𑈳 𑈠𑈴 𑈠𑈵 𑈠𑈶 𑈠𑈷 \\ 

𑈡 𑈡𑈬 𑈡𑈭 𑈡𑈮 𑈡𑈯 𑈡𑈰 𑈡𑈱 𑈡𑈲 𑈡𑈳 𑈡𑈴 𑈡𑈵 𑈡𑈶 𑈡𑈷 \\ 

𑈢 𑈢𑈬 𑈢𑈭 𑈢𑈮 𑈢𑈯 𑈢𑈰 𑈢𑈱 𑈢𑈲 𑈢𑈳 𑈢𑈴 𑈢𑈵 𑈢𑈶 𑈢𑈷 \\ 

𑈣 𑈣𑈬 𑈣𑈭 𑈣𑈮 𑈣𑈯 𑈣𑈰 𑈣𑈱 𑈣𑈲 𑈣𑈳 𑈣𑈴 𑈣𑈵 𑈣𑈶 𑈣𑈷 \\ 

𑈤 𑈤𑈬 𑈤𑈭 𑈤𑈮 𑈤𑈯 𑈤𑈰 𑈤𑈱 𑈤𑈲 𑈤𑈳 𑈤𑈴 𑈤𑈵 𑈤𑈶 𑈤𑈷 \\ 

𑈥 𑈥𑈬 𑈥𑈭 𑈥𑈮 𑈥𑈯 𑈥𑈰 𑈥𑈱 𑈥𑈲 𑈥𑈳 𑈥𑈴 𑈥𑈵 𑈥𑈶 𑈥𑈷 \\ 

𑈦 𑈦𑈬 𑈦𑈭 𑈦𑈮 𑈦𑈯 𑈦𑈰 𑈦𑈱 𑈦𑈲 𑈦𑈳 𑈦𑈴 𑈦𑈵 𑈦𑈶 𑈦𑈷 \\ 

𑈧 𑈧𑈬 𑈧𑈭 𑈧𑈮 𑈧𑈯 𑈧𑈰 𑈧𑈱 𑈧𑈲 𑈧𑈳 𑈧𑈴 𑈧𑈵 𑈧𑈶 𑈧𑈷 \\ 

𑈨 𑈨𑈬 𑈨𑈭 𑈨𑈮 𑈨𑈯 𑈨𑈰 𑈨𑈱 𑈨𑈲 𑈨𑈳 𑈨𑈴 𑈨𑈵 𑈨𑈶 𑈨𑈷 \\ 

𑈩 𑈩𑈬 𑈩𑈭 𑈩𑈮 𑈩𑈯 𑈩𑈰 𑈩𑈱 𑈩𑈲 𑈩𑈳 𑈩𑈴 𑈩𑈵 𑈩𑈶 𑈩𑈷 \\ 

𑈩𑈶 𑈩𑈶𑈬 𑈩𑈶𑈭 𑈩𑈶𑈮 𑈩𑈶𑈯 𑈩𑈶𑈰 𑈩𑈶𑈱 𑈩𑈶𑈲 𑈩𑈶𑈳 𑈩𑈶𑈴 𑈩𑈶𑈵 𑈩𑈶𑈶 𑈩𑈷𑈶 \\ 

𑈪 𑈪𑈬 𑈪𑈭 𑈪𑈮 𑈪𑈯 𑈪𑈰 𑈪𑈱 𑈪𑈲 𑈪𑈳 𑈪𑈴 𑈪𑈵 𑈪𑈶 𑈪𑈷 \\ 

𑈫 𑈫𑈬 𑈫𑈭 𑈫𑈮 𑈫𑈯 𑈫𑈰 𑈫𑈱 𑈫𑈲 𑈫𑈳 𑈫𑈴 𑈫𑈵 𑈫𑈶 𑈫𑈷 \\ 
}
\bigskip


The following are consonant conjuncts produced according to the 
virama model:
\bigskip


% kssa
<ka, virama, sa, nukta, vowel sign ...> \\
{\Large\khoj 
𑈈𑈵𑈩𑈶 𑈈𑈵𑈩𑈶𑈬 𑈈𑈵𑈩𑈶𑈭 𑈈𑈵𑈩𑈶𑈮 𑈈𑈵𑈩𑈶𑈯 𑈈𑈵𑈩𑈶𑈰 𑈈𑈵𑈩𑈶𑈱 𑈈𑈵𑈩𑈶𑈲 𑈈𑈵𑈩𑈶𑈳 𑈈𑈵𑈩𑈶𑈴 𑈈𑈵𑈩𑈶𑈵 𑈈𑈵𑈩𑈶𑈶 𑈈𑈵𑈩𑈶𑈷 \\
}

% jnya
<ja, virama, nya> \\
{\Large\khoj 
𑈐𑈵𑈓 𑈐𑈵𑈓𑈬 𑈐𑈵𑈓𑈭 𑈐𑈵𑈓𑈮 𑈐𑈵𑈓𑈯 𑈐𑈵𑈓𑈰 𑈐𑈵𑈓𑈱 𑈐𑈵𑈓𑈲 𑈐𑈵𑈓𑈳 𑈐𑈵𑈓𑈴 𑈐𑈵𑈓𑈵 𑈐𑈵𑈓𑈶 𑈐𑈵𑈓𑈷 \\
}

% tra
<ta, virama, ra> \\
{\Large\khoj 
𑈙𑈵𑈦 𑈙𑈵𑈦𑈬 𑈙𑈵𑈦𑈭 𑈙𑈵𑈦𑈮 𑈙𑈵𑈦𑈯 𑈙𑈵𑈦𑈰 𑈙𑈵𑈦𑈱 𑈙𑈵𑈦𑈲 𑈙𑈵𑈦𑈳 𑈙𑈵𑈦𑈴 𑈙𑈵𑈦𑈵 𑈙𑈵𑈦𑈶 𑈙𑈵𑈦𑈷 \\ 
}

% dra
<da, virama, ra> \\
{\Large\khoj 
𑈛𑈵𑈦 𑈛𑈵𑈦𑈬 𑈛𑈵𑈦𑈭 𑈛𑈵𑈦𑈮 𑈛𑈵𑈦𑈯 𑈛𑈵𑈦𑈰 𑈛𑈵𑈦𑈱 𑈛𑈵𑈦𑈲 𑈛𑈵𑈦𑈳 𑈛𑈵𑈦𑈴 𑈛𑈵𑈦𑈵 𑈛𑈵𑈦𑈶 𑈛𑈵𑈦𑈷 \\ 
}
\bigskip

Here are examples showing the usage of anusvara, nukta, shadda 
with vowel signs:
\bigskip

{\Large\khoj 
𑈈𑈶 𑈈𑈶𑈬 𑈈𑈶𑈭 𑈈𑈶𑈮 𑈈𑈶𑈯 𑈈𑈶𑈰 𑈈𑈶𑈱 𑈈𑈶𑈲 𑈈𑈶𑈳 𑈈𑈶𑈴 𑈈𑈶𑈵 𑈈𑈶𑈶 𑈈𑈶𑈷 \\

𑈈𑈷 𑈈𑈷𑈬 𑈈𑈷𑈭 𑈈𑈷𑈮 𑈈𑈷𑈯 𑈈𑈷𑈰 𑈈𑈷𑈱 𑈈𑈷𑈲 𑈈𑈷𑈳 𑈈𑈷𑈴 𑈈𑈷𑈵 𑈈𑈷𑈶 𑈈𑈶𑈷 \\

𑈈𑈴 𑈈𑈬𑈴 𑈈𑈭𑈴 𑈈𑈮𑈴 𑈈𑈯𑈴 𑈈𑈰𑈴 𑈈𑈱𑈴 𑈈𑈲𑈴 𑈈𑈳𑈴 \\

𑈈𑈶𑈴 𑈈𑈶𑈬𑈴 𑈈𑈶𑈭𑈴 𑈈𑈶𑈮𑈴 𑈈𑈶𑈯𑈴 𑈈𑈶𑈰𑈴 𑈈𑈶𑈱𑈴 𑈈𑈶𑈲𑈴 𑈈𑈶𑈳𑈴 𑈈𑈶𑈴 𑈈𑈶𑈶𑈴 𑈈𑈶𑈷𑈴 \\

𑈈𑈷𑈴 𑈈𑈷𑈬𑈴 𑈈𑈷𑈭𑈴 𑈈𑈷𑈮𑈴 𑈈𑈷𑈯𑈴 𑈈𑈷𑈰𑈴 𑈈𑈷𑈱𑈴 𑈈𑈷𑈲𑈴 𑈈𑈷𑈳𑈴 𑈈𑈷𑈴 𑈈𑈷𑈶𑈴 𑈈𑈶𑈷𑈴 \\
}
\bigskip

Below is an excerpt of a \textit{ginan} in Khojki: 
\bigskip

{\large\khoj 
𑈁𑈄 𑈦𑈪𑈰𑈤 𑈦𑈪𑈰𑈤𑈬𑈞 𑈀𑈢 𑈙𑈲 𑈦𑈪𑈰𑈤 𑈈𑈦𑈲𑈴𑈊𑈰 𑈸 \\
𑈄𑈐𑈮 𑈙𑈞 𑈤𑈞 𑈛𑈞 𑈊𑈯𑈦𑈯 𑈀𑈦𑈟𑈘 𑈈𑈮𑈐𑈰 𑈸 \\
𑈙𑈲 𑈊𑈭𑈞𑈬𑈞𑈰 𑈊𑈭𑈞𑈬𑈞𑈰 𑈊𑈭𑈞𑈬𑈞 𑈀𑈢 𑈙𑈲 𑈹 \\
𑈜𑈬𑈞 𑈩𑈉𑈬𑈨𑈙 𑈪𑈦𑈛𑈤 𑈈𑈮𑈐𑈰 𑈸 \\
𑈙𑈲 𑈜𑈬𑈞𑈰 𑈜𑈬𑈞𑈰 𑈜𑈬𑈞 𑈀𑈢 𑈙𑈲 𑈹 \\
𑈄𑈐𑈮 𑈩𑈡 𑈊𑈔 𑈄𑈈𑈐 𑈦𑈪𑈰𑈤𑈬𑈞 𑈜𑈮𑈩𑈰 𑈸 \\ 
𑈙𑈲 𑈩𑈶𑈬𑈞𑈰 𑈩𑈶𑈬𑈞𑈰 𑈩𑈶𑈬𑈞 𑈀𑈢 𑈙𑈲 𑈹 \\
𑈄𑈐𑈮 𑈈𑈪𑈰𑈙 𑈂𑈤𑈬𑈤 𑈢𑈰𑈊𑈤 𑈤𑈰𑈦𑈬 𑈟𑈮𑈦 𑈪𑈩𑈞 𑈩𑈶𑈬𑈪𑈬 𑈸 \\
𑈂𑈤𑈬𑈞𑈰 𑈂𑈤𑈬𑈞𑈰 𑈂𑈤𑈬𑈞 𑈀𑈡 𑈙𑈲 𑈹 𑈼 \\
}

Below is a version of the above with the word 
separator used instead of spaces:
\bigskip

{\large\khoj 
𑈁𑈄𑈺𑈦𑈪𑈰𑈤𑈺𑈦𑈪𑈰𑈤𑈬𑈞𑈺𑈀𑈢𑈺𑈙𑈲𑈺𑈦𑈪𑈰𑈤𑈺𑈈𑈦𑈲𑈴𑈊𑈰 𑈸 \\
𑈄𑈐𑈮𑈺𑈙𑈞𑈺𑈤𑈞𑈺𑈛𑈞𑈺𑈊𑈯𑈦𑈯𑈺𑈀𑈦𑈟𑈘𑈺𑈈𑈮𑈐𑈰 𑈸 \\
𑈙𑈲𑈺𑈊𑈭𑈞𑈬𑈞𑈰𑈺𑈊𑈭𑈞𑈬𑈞𑈰𑈺𑈊𑈭𑈞𑈬𑈞𑈺𑈀𑈢𑈺𑈙𑈲 𑈹 \\
𑈜𑈬𑈞𑈺𑈩𑈉𑈬𑈨𑈙𑈺𑈪𑈦𑈛𑈤𑈺𑈈𑈮𑈐𑈰 𑈸 \\
𑈙𑈲𑈺𑈜𑈬𑈞𑈰𑈺𑈜𑈬𑈞𑈰𑈺𑈜𑈬𑈞𑈺𑈀𑈢𑈺𑈙𑈲 𑈹 \\
𑈄𑈐𑈮𑈺𑈩𑈡𑈺𑈊𑈔𑈺𑈄𑈈𑈐𑈺𑈦𑈪𑈰𑈤𑈬𑈞𑈺𑈜𑈮𑈩𑈰 𑈸 \\
𑈙𑈲𑈺𑈩𑈶𑈬𑈞𑈰𑈺𑈩𑈶𑈬𑈞𑈰𑈺𑈩𑈶𑈬𑈞𑈺𑈀𑈢𑈺𑈙𑈲 𑈹 \\
𑈄𑈐𑈮𑈺𑈈𑈪𑈰𑈙𑈺𑈂𑈤𑈬𑈤𑈺𑈢𑈰𑈊𑈤𑈺𑈤𑈰𑈦𑈬𑈺𑈟𑈮𑈦𑈺𑈪𑈩𑈞𑈺𑈩𑈶𑈬𑈪𑈬 𑈸 \\
𑈂𑈤𑈬𑈞𑈰𑈺𑈂𑈤𑈬𑈞𑈰𑈺𑈂𑈤𑈬𑈞𑈺𑈀𑈡𑈺𑈙𑈲 𑈹 𑈼 \\
}


\end{document}
